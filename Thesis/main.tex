\documentclass[12pt]{article}
\usepackage{geometry}
\usepackage{stmaryrd}
\usepackage{imakeidx}
\geometry{left=1in,right=0.75in,top=1in,bottom=1in}

%%%%%%%%%%%%%%%%%%%%%%%%%%%%%%%%%%%%%%%%
% Replace ABCDEF in the next line with your chosen problem
% and replace 1111111 with your Team Control Number
\newcommand{\Problem}{ABCDEF}
\newcommand{\Team}{2503720}
%%%%%%%%%%%%%%%%%%%%%%%%%%%%%%%%%%%%%%%%
\usepackage[backend=bibtex]{biblatex}

\usepackage{newtxtext}
\usepackage{amsmath,amssymb,amsthm}
\usepackage{newtxmath} % must come after amsXXX
\usepackage{tocloft}

% \usepackage[pdftex]{graphicx}
\usepackage{xcolor}
\usepackage{fancyhdr}
%-----数学宏包-----
\usepackage{mathrsfs,bm}
%-----draft下 label提示-----
\usepackage[notcite,notref]{showkeys}
% %-----设置超链接-----
\usepackage{url,hyperref}
\hypersetup{colorlinks=true,linkcolor=black,citecolor=black} % 去掉目录红框
%-----制作目录-----
\usepackage{imakeidx}
% 设置颜色
\usepackage{color,xcolor}
% 插入图片
\usepackage{graphicx}
\usepackage{epsfig}
%-----设置表格-----
\usepackage{tabularx,array}
\usepackage{longtable}
\usepackage{booktabs}
\usepackage{multirow}
\usepackage{multicol}
%-----调整单元格格式-----
\usepackage{makecell}
%-----操作字符串-----
\usepackage{xstring}
%-----多语种处理-----
%\usepackage[english]{babel}
%-----设置代码环境-----
\usepackage{listings}
%-----设置章节标题和目录-----
\usepackage{titling}
\usepackage{titletoc}
\usepackage{titlesec}
%-----数学公式扩展-----
\usepackage{mathtools}
%-----浮动体设置-----
\usepackage{float}
%-----书签设置-----
\usepackage{bookmark}
%-----数学宏包-----
\usepackage{amsmath,amsthm,amssymb,amsfonts}
\usepackage{mathrsfs,bm}
%-----draft下 label提示-----
\usepackage[notcite,notref]{showkeys}
% --- 算法宏包及设置 ---
\usepackage{algorithm}
\usepackage{algpseudocode}
% ---- 定义列表项的样式 -----
\usepackage{enumitem}
\setlist{nolistsep}
% --- 设置英文字体 -----
\usepackage{newtxtext}  % for text fonts
% --- 设置数学字体 -----
\usepackage{newtxmath}
% --- 直接插入 pdf 文件 ----
\usepackage{pdfpages}
\usepackage{lastpage} % 用于获取总页数
\usepackage[titletoc,title]{appendix} % 用于添加附录
\usepackage{setspace} % 用于设置行间距
% 插入图片
\usepackage{graphicx}
\usepackage{epsfig}
%-----设置表格-----
\usepackage{tabularx,array}
\usepackage{longtable}
\usepackage{booktabs}
\usepackage{multirow}
\usepackage{multicol}
%-----调整单元格格式-----
\usepackage{makecell}
%-----操作字符串-----
\usepackage{xstring}
%-----多语种处理-----
%\usepackage[english]{babel}
%-----设置代码环境-----
\usepackage{listings}
%-----设置章节标题和目录-----
\usepackage{titletoc}
\usepackage{titlesec}
%-----数学公式扩展-----
\usepackage{mathtools}
%-----浮动体设置-----
\usepackage{float}
%-----书签设置-----
\usepackage{bookmark}
\usepackage{algorithm}
\usepackage{algpseudocode}
\usepackage{setspace}
\usepackage{minted} % 用于插入代码
\renewcommand*{\baselinestretch}{1.5}
\newcommand{\upcite}[1]{\textsuperscript{\textsuperscript{\cite{#1}}}}
\definecolor{bg}{rgb}{0.95, 0.95, 0.95}
% 设置 minted 环境的全局选项
\setminted{
    linenos, % 显示行号
    frame=lines, % 添加边框
    framesep=2mm, % 设置边框与代码之间的间距
    breaklines, % 自动换行
    bgcolor=bg, % 设置背景颜色
    baselinestretch = 1.0,
    numbersep = 5pt,
    numberblanklines = false,
	tabsize = 4
}
% ----- 设置浮动体间距 ------
\setlength{\textfloatsep}{0pt}
\setlength{\floatsep}{10pt plus 3pt minus 2pt}
\setlength{\intextsep}{10pt}
\setlength{\abovecaptionskip}{2pt plus1pt minus1pt}
\setlength{\belowcaptionskip}{3pt plus1pt minus2pt}

% ----- 设置公式间距为零 ------
\AtBeginDocument{
	\setlength{\abovedisplayskip}{4pt plus1pt minus1pt}
	\setlength{\belowdisplayskip}{4pt plus1pt minus1pt}
	\setlength{\abovedisplayshortskip}{2pt}
	\setlength{\belowdisplayshortskip}{2pt}
	\setlength{\arraycolsep}{2pt}   % array中列之间空白长度
}
% ---- 定义列表项的样式 -----
\usepackage{enumitem}
\setlist{nolistsep}
% \setlength{\itemsep}{3pt plus1pt minus2pt}

% --- 设置英文字体 -----
% \usepackage{newtxtext}  % for text fonts
\usepackage{fontspec}


% --- 自定义命令 -----
\newcommand{\CC}{\ensuremath{\mathbb{C}}}
\newcommand{\RR}{\ensuremath{\mathbb{R}}}
\newcommand{\A}{\mathcal{A}}
\newcommand{\ii}{\bm{\mathrm{i}}\,}  % 虚部
\newcommand{\md}{\mathrm{d}\,}
\newcommand{\bA}{\boldsymbol{A}}
\newcommand{\red}[1]{\textcolor{red}{#1}}


\lhead{Team \Team}
\rhead{}
\cfoot{}

\newtheorem{theorem}{Theorem}
\newtheorem{corollary}[theorem]{Corollary}
\newtheorem{lemma}[theorem]{Lemma}
\newtheorem{definition}{Definition}

% 定义 summary 环境
\newenvironment{summary}
  {\begin{center}\bfseries Summary\end{center}\normalfont}

% 重新定义目录标题并居中显示
\renewcommand{\contentsname}{\centering Contents}

% 设置页眉
\pagestyle{fancy}
\fancyhf{}
\fancyhead[L]{Team \# 1234567}
\fancyhead[R]{Page \thepage\ of \pageref{LastPage}}
%%%%%%%%%%%%%%%%%%%%%%%%%%%%%%%%




\begin{document}

\onehalfspacing

\thispagestyle{empty}
\vspace*{-16ex}
\centerline{\begin{tabular}{*3{c}}
	\parbox[t]{0.3\linewidth}{\begin{center}\textbf{Problem Chosen}\\ \Large \textcolor{red}{\Problem}\end{center}}
	& \parbox[t]{0.3\linewidth}{\begin{center}\textbf{2025\\ MCM / ICM\\ Summary Sheet}\end{center}}
	& \parbox[t]{0.3\linewidth}{\begin{center}\textbf{Team Control Number}\\ \Large \textcolor{red}{\Team}\end{center}}	\\
	\bottomrule
\end{tabular}}

\begin{center}
  \Large{\textbf{This is the title}}
\end{center}
%%%%%%%%%% Begin Summary %%%%%%%%%%%
\begin{summary}
Here is the abstract of our paper.
\end{summary}

\clearpage

%%%%%%%%%% End Summary %%%%%%%%%%%

\tableofcontents

%%%%%%%%%%%%%%%%%%%%%%%%%%%%%
\clearpage
\pagestyle{fancy}
% Uncomment the next line to generate a Table of Contents
%\tableofcontents
\newpage

%%%%%%%%%%%%%%%%%%%%%%%%%%%%%%

\section{Introduction}

\subsection{Problem Background}

\begin{itemize}
    \item First
    \item Second
\end{itemize}

\subsection{Literature Review}

\subsubsection{Whatever}

%%%%%%%%%%%%%%%%%%%%%%%%%%%%%%

%引用lastpage宏包,用于获取总页数

\section{Preparations of the Models}

\subsection{Assumptions}

\subsection{Notations}

The primary notations used in this paper are listed in Table 1.

\begin{table}[!htbp]
  \begin{center}
  \caption{Notations}
  \begin{tabular}{cl}
    \toprule
    \multicolumn{1}{m{3cm}}{\centering Symbol}
    &\multicolumn{1}{m{8cm}}{\centering Definition}\\
    \midrule
    $A$&the first one\\
    $b$&the second one\\
    $\alpha$ &the last one\\
    \bottomrule
  \end{tabular}
  \end{center}
  \end{table}


\section{The Models}

\subsection{Model 1}

\subsubsection{Details about Model 1}

The datail can be described as follows:
\begin{equation}
  \frac{\partial u}{\partial t}-a^2\left(\frac{\partial^2 u}{\partial x^2}+\frac{\partial^2 u}{\partial y^2}+\frac{\partial^2 u}{\partial z^2}\right)=f(x, y, z, t)
\end{equation}

\clearpage
\addcontentsline{toc}{section}{References}

\begin{thebibliography}{99}
  \bibitem{} Einstein, A., Podolsky, B., \& Rosen, N. (1935). Can quantum-mechanical description of physical reality be considered complete?. \emph{Physical review}, 47(10), 777.
  \bibitem{} \emph{A simple, easy \LaTeX\ template for MCM/ICM: EasyMCM}. (2018). Retrieved December 1, 2019
\end{thebibliography}
\clearpage

\begin{appendices}
  \section{Further on \LaTeX}
  \section{Program Codes}
  \begin{minted}{cpp}
#include <iostream>
using namespace std;
int main() {
    cout << "Hello, World!" << endl;
    return 0;
}
  \end{minted}
\end{appendices}
    



\end{document}