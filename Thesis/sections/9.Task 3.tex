\section{Task 3: Memo}
\begin{center}
    \Large\textbf{Recommendations for Sustainable Tourism in Juneau, Alaska}
\end{center}

{

Dear Tourist Council,

It is our great honour to present to you our recommendations for 
sustainable tourism in Juneau, Alaska. We have conducted a thorough 
analysis of the current situation in Juneau and have identified several key 
areas that need to be addressed in order to ensure the long-term sustainability 
of the tourism industry in the region. 
Our recommendations are based on the principles of sustainable tourism, 
which aim to balance the economic, social, and environmental impacts of tourism 
in order to ensure that it can continue to benefit both the local community and 
the environment for generations to come. Our approaches, findings and suggestions are as follows.

Firstly we summarized a general equation aiming to balance the economic, social, 
and environmental impacts of tourism. Then we looked into these aspects
and devised a model accordingly for each. \textit{SARIMAX, Linear-Regression} models 
were used to ensure the accuracy and reliability of our findings and suggestions.

Here are some findings based on our predictions.

\begin{itemize}
    \item The emission of carbon dioxide from tourism is increasing at an alarming rate, and it is 
    highly correlated with the square number of tourists.
    \item Current tax rates may not be optimal for maximizing tourism income and environmental sustainability.
    \item The number of tourists is higher than the optimal number that can be accommodated by the local environment.
\end{itemize}

Based on these findings, we put forward the following recommendations and measures.

\begin{itemize}
    \item Impose a carbon tax of around 20\% on tourism to reduce carbon emissions and encourage sustainable practices.
    \item Increase the fine amount to 20\$ for tourists who violate environmental regulations to deter harmful behavior.
    \item Set the upper limit of tourists to 1.5 million per year to protect the local environment and culture. Reduce the number of tourists by 10\% each year to around 1.2 million by 2030.
\end{itemize}

I hope you find our recommendations useful and that they will help to guide
the development of sustainable tourism in Juneau. We believe that by working
together, we can create a more sustainable future for the tourism industry in
the region. Thank you for your attention and consideration. Feel free to contact us for any further information.

\begin{flushright}
    Sincerely,\\
    Team \# 2503720 Members
    \end{flushright}

}

\clearpage