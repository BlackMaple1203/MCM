\section{Task 1: Model for Tourism Industry in Juneau}

\subsection{Introduction}


In this section we need to select factors to quantify and track the tourism industry in Juneau. 
It is impossible and unnecessary to consider all the factors that may affect the 
tourism industry, only those that are relevant to the problem need to be considered.
Drawing on the idea of the divide-and-conquer algorithm, we first divide the factors 
into three categories: economy, society and environment. 


\begin{equation}
    \mathbb{T}\text { Output }=\alpha \cdot \text { Economy }-\beta \cdot \text { Society }-\gamma \cdot \text { Environment }
\end{equation}

Our goal is to maximize the economy income, minimize the social cost and environmental impact,
where parameters $\alpha$, $\beta$ and $\gamma$ denote how much importance we attach to each category.
Intuitively, the goal aforementioned is equivalent to maximizing the output.

Each category is further divided into several minor factors such as local population, 
number of tourists to extrapolate a mathematical model fitting the circumstances in Juneau,
which will be discussed in the following sections.

\subsection{Preliminary Analyses}

We first analyse the potential factors that may affect the tourism industry in Juneau, thus enabling
 a smoother transition to the model building process.

 \subsubsection{Number of Tourists}

 We found no existing data on the number of tourists visiting Juneau each year, but we can infer it
 by other means.

 According to [source] and [source], among all the transportation methods, cruise ships are the most popular way to visit Juneau,
 accounting for over 90\% of the total number of tourists. As the number of cruise ship passengers is available online, 
 we can use it as a proxy to estimate the total number of tourists.

 According to [source], the number of cruise ship passengers visiting Juneau is as follows:

 \begin{table}[H]
    \centering
    \renewcommand{\arraystretch}{1.3}
 \begin{tabular}{|c|c|c|c|c|c|c|c|c|c|c|}
    \hline \textit{Year} & 2014 & 2015 & 2016 & 2017 & 2018 & 2019 & 2020 & 2021 & 2022 & 2023 \\
    \hline \begin{tabular}{l} 
    \textit{Num(in thousands)} 
    \end{tabular} & 961 & 983 & 1015 & 1072 & 1151 & 1306 & 0 & 117 & 1167 & 1670 \\
    \hline
    \end{tabular}
    \caption{Number of Cruise Ship Visitors to Juneau}
\end{table}

\subsection{Economy}

In this section we consider the actions that will contribute to the income of the tourism industry in Juneau, which are
tourists' consumption, tax income and fines.

\subsubsection{Tourists' Consumption}


\subsection{Society}

Societal factors such as infrastructure, price of housing products, and the mental
loss due to the overcrowding and rowdy tourists all account for the social cost of the tourism industry.

\subsection{Environment}

According to the official website of Juneau, its tourism industry is 
mainly comprised of glacier tours, whale watching, rainforest tours and others.
We assume each of these activities accounts for a certain percentage of the total environmental impact,
denoted as $v_1$, $v_2$, $v_3$ and $v_4$ respectively. Due to the receding of glaciers,
our goal is to lower the percentage of glacier tours and increase the percentage of other activities.

The main factors that affect the environment are carbon emissions and human disturbance, which will be discussed as follows.